\chapter{Protection}
\label{chap-protection}

There are two kinds of protection that are important in an operating
system:

\begin{itemize}
\item \emph{protecting different users from each other}.  User A
  should not be able to access or destroy the data of some other user
  B, other than if B explicitly permits it, and then only in ways that
  are acceptable to B.
\item \emph{protecting the system from the users}.  Users should be
  able to access system resources such as memory and peripherals only
  in controlled ways, so as to guarantee the integrity of the system.
\end{itemize}

\section{Protecting users from each other}

We use a combination of \emph{access control lists} and
\emph{capabilities}.  All heap-allocated objects except \texttt{cons}
cells and (heap-allocated) numbers are manipulated through a
\emph{tagged pointer}.  In addition to containing a type tag, the
pointer also contains an \emph{access tag}.  The access tag consists
of the 4 most-significant bits of a 64-bit pointer.  Before a pointer
is used to fetch an object from memory, the access bits are cleared.
A primitive operation to fetch the access tag of a pointer is
available to any user code.  Each of the 4 bits represents a potential
\emph{access restriction}, the significance of which is up to the
programmer.  A function that wishes to restrict permission to some
object can test the corresponding access bit and signal an error if
that bit is set.

The author of some complex data structure may for instance grant
access to it only to certain other users.  This would be done by
interpreting one of the access bits as \emph{read permission}, and by
having generic functions that access the data structures check that
this bit has the desired value (for instance in a \texttt{:before}
method). 

The access bits of a capability are determined when the object is
accessed through the object store.  \seechap{chap-object-store} One of
the possible attributes associated with the object in the object store
corresponds to the access permissions in the form of an \emph{access
  control list}.  A user who accesses the object from the object store
will be checked against the access control list and appropriate access
bits will be cleared in the object before it is given to the user.

\section{Protecting the system from the users}

In a typical modern operating system, the system is protected from the
users through the use of a \emph{mode} of execution of the processor,
which can be either \emph{user mode} or \emph{supervisor mode}.
Certain instructions are restricted to supervisor mode, such as
instructions for input/output or for remapping the address space. 

In \sysname{}, the normal mode of execution is \emph{supervisor mode}.
The code executed by the user is translated to machine code by a
compiler which is known not to generate code that, if executed, might
represent a risk to the integrity of the system.  Since no remapping
of the address space is required as a result of an \emph{interrupt} or
a \emph{trap}, such events can be handled very quickly.

Occasionally, it might be useful to write or install some software
that is compiled to machine code by some compiler that does not
necessarily generate code with controlled access, such as a compiler
for some typical low-level programming language used today.  The
result of such a compilation or installation is a single (possibly
large) Lisp function.  When this function is executed, the mode of
execution is switched to \emph{user mode}.  As with traditional modern
operating systems, the code of such software has its own \emph{address
  space}, which means that it can not directly manipulate \sysname{}
capabilities.  Instead, it has to communicate with the system through
the user of \emph{system calls}.  A system-wide object is referred to
by such code through an interposing \emph{object descriptor}, much
like a file descriptor in \unix{}.  The details of this mechanism have
not yet been fully determined.


